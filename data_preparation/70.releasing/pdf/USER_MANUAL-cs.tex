% Options for packages loaded elsewhere
\PassOptionsToPackage{unicode}{hyperref}
\PassOptionsToPackage{hyphens}{url}
%
\documentclass[
]{article}
\usepackage{lmodern}
\usepackage{amssymb,amsmath}
\usepackage{ifxetex,ifluatex}
\ifnum 0\ifxetex 1\fi\ifluatex 1\fi=0 % if pdftex
  \usepackage[T1]{fontenc}
  \usepackage[utf8]{inputenc}
  \usepackage{textcomp} % provide euro and other symbols
\else % if luatex or xetex
  \usepackage{unicode-math}
  \defaultfontfeatures{Scale=MatchLowercase}
  \defaultfontfeatures[\rmfamily]{Ligatures=TeX,Scale=1}
\fi
% Use upquote if available, for straight quotes in verbatim environments
\IfFileExists{upquote.sty}{\usepackage{upquote}}{}
\IfFileExists{microtype.sty}{% use microtype if available
  \usepackage[]{microtype}
  \UseMicrotypeSet[protrusion]{basicmath} % disable protrusion for tt fonts
}{}
\makeatletter
\@ifundefined{KOMAClassName}{% if non-KOMA class
  \IfFileExists{parskip.sty}{%
    \usepackage{parskip}
  }{% else
    \setlength{\parindent}{0pt}
    \setlength{\parskip}{6pt plus 2pt minus 1pt}}
}{% if KOMA class
  \KOMAoptions{parskip=half}}
\makeatother
\usepackage{xcolor}
\IfFileExists{xurl.sty}{\usepackage{xurl}}{} % add URL line breaks if available
\IfFileExists{bookmark.sty}{\usepackage{bookmark}}{\usepackage{hyperref}}
\hypersetup{
  hidelinks,
  pdfcreator={LaTeX via pandoc}}
\urlstyle{same} % disable monospaced font for URLs
\setlength{\emergencystretch}{3em} % prevent overfull lines
\providecommand{\tightlist}{%
  \setlength{\itemsep}{0pt}\setlength{\parskip}{0pt}}
\setcounter{secnumdepth}{-\maxdimen} % remove section numbering

\author{}
\date{}

\begin{document}

\hypertarget{databuxe1ze-mluvenuxfdch-projevux16f-v-ux10deux161tinux11b-jako-cizuxedm-jazyce-trvaluxfd-pobyt-v-ux10dr-uux17eivatelskuxe1-pux159uxedruux10dka}{%
\section{Databáze mluvených projevů v češtině jako cizím jazyce (trvalý
pobyt v ČR): uživatelská
příručka}\label{databuxe1ze-mluvenuxfdch-projevux16f-v-ux10deux161tinux11b-jako-cizuxedm-jazyce-trvaluxfd-pobyt-v-ux10dr-uux17eivatelskuxe1-pux159uxedruux10dka}}

Základní funkce databáze zahrnuje prohlížení záznamů s různými způsoby
jejich zobrazení, filtrování záznamů podle různých kategorií a komplexní
vyhledávání v obsahu databáze. Databáze rovněž umožňuje stáhnout korpus
jako celek nebo stáhnout vybrané záznamy.

\hypertarget{prohluxedux17eenuxed-zuxe1znamux16f}{%
\subsection{Prohlížení
záznamů}\label{prohluxedux17eenuxed-zuxe1znamux16f}}

Po vstupu do korpusu se v přehledné tabulce zobrazí všechny záznamy (tj.
soubory transkriptů) uložené v databázi. Pro každý soubor s transkriptem
tabulka kromě názvu souboru zobrazuje v dalších sloupcích úroveň a
identifikátor zkoušky, číslo úlohy, zdroj předběžné anotace, kód
anotátora a informaci o tom, zda je přepis pro danou nahrávku kanonický.
Soubory v tabulce je možné třídit podle hodnot vybraného sloupce.
Záznamy lze také filtrovat na základě libovolného podřetězce v názvu
souboru zadáním tohoto podřetězce do textového pole ``Search''
umístěného vpravo nad tabulkou. Kliknutím na konkrétní soubor se tento
soubor zobrazí.

\hypertarget{zobrazenuxed-souboru}{%
\subsection{Zobrazení souboru}\label{zobrazenuxed-souboru}}

Databáze umožňuje prohlížet přepisy jednotlivých replik spolu s
anotacemi a metadaty a také poslouchat příslušné zvukové nahrávky.
Charakter zobrazených informací se liší podle zvoleného režimu
zobrazení, mezi kterými lze přepínat v dolní části stránky pod samotným
přepisem.

\hypertarget{reux17eim-text-view}{%
\subsubsection{Režim Text View}\label{reux17eim-text-view}}

Text View je základní režim zobrazení, který se objeví po otevření
souboru. V horní části obrazovky se nachází hlavička s názvem přepisu a
vybranými metadaty. V dolní části je zobrazen samotný přepis, rozdělený
na repliky. Každá replika je označena identifikátorem mluvčího (EXAM\_1
pro zkoušejícího a CAND\_1 pro kandidáta).

Tento režim rovněž umožňuje zobrazit automatickou morfologickou anotaci
a lemmatizaci. Po najetí kurzorem na konkrétní token se zobrazí
příslušná anotace v kontextu. Pro zobrazení vybraného atributu pro
všechny tokeny v přepisu lze využít ovládací prvky umístěné pod
hlavičkou, které obsahují následující tlačítka: - PoS: Zobrazí slovní
druhy. - Tag: Ukáže morfologické tagy. - Features: Poskytne podrobné
morfologické informace. - Lemma: Zobrazí základní tvary slov.

\hypertarget{reux17eim-waveform-view}{%
\subsubsection{Režim Waveform View}\label{reux17eim-waveform-view}}

V horní části obrazovky se nachází rozšířený ovládací prvek pro
přehrávání nahrávky, který zobrazuje graf signálu (tzv. waveform). Pod
ním jsou zobrazeny přepisy jednotlivých replik. Kliknutím na konkrétní
repliku se tato replika přehraje.

\hypertarget{reux17eim-dependencies}{%
\subsubsection{Režim Dependencies}\label{reux17eim-dependencies}}

Tento režim zobrazuje syntaktickou anotaci. Po kliknutí na konkrétní
repliku se zobrazí automaticky vygenerovaný závislostní strom, u nějž je
možné zobrazit detaily pomocí myši. Vpravo nahoře od stromu se nachází
tlačítko $\equiv$ pro další možnosti zobrazení stromu. Je tak možné uspořádat
uzly podle slovosledu, zobrazit interpunkci nebo uložit obrázek stromu
ve formátu SVG.

\hypertarget{filtrovuxe1nuxed-zuxe1znamux16f-pux159es-kategorie}{%
\subsection{Filtrování záznamů přes
kategorie}\label{filtrovuxe1nuxed-zuxe1znamux16f-pux159es-kategorie}}

Po kliknutí na tlačítko \emph{Kategorie} v levém hlavním menu je možné
filtrovat přepisy na základě hodnot jednotlivých kategorií. Například je
tak možné zobrazit si pouze seznam kanonických přepisů nebo přepisů od
konkrétního anotátora.

\hypertarget{vyhleduxe1vuxe1nuxed}{%
\subsection{Vyhledávání}\label{vyhleduxe1vuxe1nuxed}}

Vyhledávání v korpusu lze provádět na stránce, která se zobrazí po
klinutí na tlačítko \emph{Hledat} v levém hlavním menu. Stránka umožňuje
zadávat dotazy ve formátu CQL (Corpus Query Language). Např.

\begin{quote}
\texttt{{[}upos\ =\ "NUM.*"{]}\ {[}lemma\ =\ "otázka"{]}}

pro nalezení tvarů slova \emph{otázka}, jimž předchází číslovka.
\end{quote}

Pro usnadnění vyhledávání nabízí rozhraní TEITOK nástroj pro sestavování
dotazů. Tento nástroj umožňuje snadno definovat jednoduché dotazy v CQL
prostřednictvím formuláře. Stačí kliknout na ikonu \emph{Query builder},
definovat svůj dotaz a poté stisknout tlačítko \emph{Create query}, čímž
se dotaz vloží do textového pole CQL, kde jej můžete případně upravit.

V základním nastavení TEITOK provádí vyhledávání v celém korpusu, který
může obsahovat k jedné nahrávce více přepisů. Pokud chcete vyhledávat
pouze v té části korpusu, v níž je ke každé nahrávce přiřazený jen
jediný přepis, je nutné omezit hledání na tzv. kanonické přepisy. Např.

\begin{quote}
\texttt{{[}lemma\ =\ "situace"{]}\ ::\ match.text\_canonical\ =\ "1"}

vyhledává lemma \emph{situace} jenom v kanonických přepisech.
\end{quote}

\hypertarget{stahovuxe1nuxed}{%
\subsection{Stahování}\label{stahovuxe1nuxed}}

Celý korpus včetně nahrávek a dokumentace je možné stáhnout z hlavního
menu vlevo.

Konkrétní přepis lze stáhnout v režimu \emph{Text view} kliknutím na
tlačítko \emph{Download XML} umístěné v dolní části stránky.

\hypertarget{jak-citovat}{%
\subsection{Jak citovat}\label{jak-citovat}}

Rysová Kateřina, Novák Michal, Rysová Magdaléna, Polák Peter, Bojar
Ondřej: \emph{Databáze mluvených projevů v češtině jako cizím jazyce
(trvalý pobyt v ČR)}. Ústav formální a aplikované lingvistiky MFF UK,
Praha 2024. Dostupná z WWW
\url{https://lindat.mff.cuni.cz/services/teitok-live/evaldio/cs/index.php?action=db_residency}.

\hypertarget{dedikace}{%
\subsection{Dedikace}\label{dedikace}}

Vznik databáze byl financován z prostředků Programu na podporu
aplikovaného výzkumu v oblasti národní a kulturní identity na léta 2023
až 2030 (NAKI III) Ministerstva kultury ČR v rámci projektu
\emph{Automatické hodnocení mluveného projevu v češtině}
(DH23P03OVV037).

\end{document}
