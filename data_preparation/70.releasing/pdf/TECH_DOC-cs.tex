% Options for packages loaded elsewhere
\PassOptionsToPackage{unicode}{hyperref}
\PassOptionsToPackage{hyphens}{url}
%
\documentclass[
]{article}
\usepackage{lmodern}
\usepackage{amssymb,amsmath}
\usepackage{ifxetex,ifluatex}
\ifnum 0\ifxetex 1\fi\ifluatex 1\fi=0 % if pdftex
  \usepackage[T1]{fontenc}
  \usepackage[utf8]{inputenc}
  \usepackage{textcomp} % provide euro and other symbols
\else % if luatex or xetex
  \usepackage{unicode-math}
  \defaultfontfeatures{Scale=MatchLowercase}
  \defaultfontfeatures[\rmfamily]{Ligatures=TeX,Scale=1}
\fi
% Use upquote if available, for straight quotes in verbatim environments
\IfFileExists{upquote.sty}{\usepackage{upquote}}{}
\IfFileExists{microtype.sty}{% use microtype if available
  \usepackage[]{microtype}
  \UseMicrotypeSet[protrusion]{basicmath} % disable protrusion for tt fonts
}{}
\makeatletter
\@ifundefined{KOMAClassName}{% if non-KOMA class
  \IfFileExists{parskip.sty}{%
    \usepackage{parskip}
  }{% else
    \setlength{\parindent}{0pt}
    \setlength{\parskip}{6pt plus 2pt minus 1pt}}
}{% if KOMA class
  \KOMAoptions{parskip=half}}
\makeatother
\usepackage{xcolor}
\IfFileExists{xurl.sty}{\usepackage{xurl}}{} % add URL line breaks if available
\IfFileExists{bookmark.sty}{\usepackage{bookmark}}{\usepackage{hyperref}}
\hypersetup{
  hidelinks,
  pdfcreator={LaTeX via pandoc}}
\urlstyle{same} % disable monospaced font for URLs
\setlength{\emergencystretch}{3em} % prevent overfull lines
\providecommand{\tightlist}{%
  \setlength{\itemsep}{0pt}\setlength{\parskip}{0pt}}
\setcounter{secnumdepth}{-\maxdimen} % remove section numbering

\author{}
\date{}

\begin{document}

\hypertarget{technickuxe1-dokumentace}{%
\section{Technická dokumentace}\label{technickuxe1-dokumentace}}

Jazykový korpus mluvených projevů nerodilých mluvčích češtiny zaměřený
na jazykovou úroveň A2 (podle SERR), požadovanou pro udělení trvalého
pobytu v České republice, je výsledkem projektu realizovaného v Ústavu
formální a aplikované lingvistiky Matematicko-fyzikální fakulty
Univerzity Karlovy. Korpus obsahuje nahrávky zaznamenávající ústní část
\href{http://ujop.cuni.cz/cce}{Certifikované zkoušky z češtiny pro
cizince} na úrovni A2. Nahrávky zahrnují dialogy mezi zkoušejícím
(rodilým mluvčím) a kandidátem zkoušky (nerodilým mluvčím). Náhravky
jsme opatřili jejich přepisy a bohatou lingvistickou anotací. K některým
nahrávkám je připojeno více přepisů od různých anotátorů, což umožňuje
srovnání různých přepisů téže nahrávky a vyhodnocení míry shody při
převodu mluvené řeči do psaného textu.

Korpus je zveřejněn jako specializovaná veřejná databáze a je volně
dostupný široké veřejnosti, vědecké komunitě, pedagogům a studentům.
Databáze je integrována do systému TEITOK, který je spravován na
platformě \href{https://lindat.cz/}{LINDAT/CLARIAH-CZ}.

\hypertarget{teitok}{%
\subsection{TEITOK}\label{teitok}}

\href{http://teitok.corpuswiki.org/}{TEITOK} je framework pro vytváření,
správu a zveřejňování anotovaných korpusů. Jeho webové rozhraní je
implementováno v kombinaci jazyků PHP a JavaScript. Pro náš projekt,
který kombinuje nahrávky mluveného projevu a jejich přepisy, je stěžejní
funkcionalita prostředí TEITOK, která umožňuje
\href{http://www.teitok.org/index.php?action=help\&id=wavesurfer}{vytvářet,
zobrazovat a upravovat přepisy nahrávek}. K práci se samotnou nahrávkou
TEITOK využívá Javascript knihovnu
\href{http://wavesurfer-js.org/}{wavesurfer}.

\hypertarget{uloux17eenuxed-dat}{%
\subsubsection{Uložení dat}\label{uloux17eenuxed-dat}}

Data korpusu jsou v prostředí TEITOK primárně uložena ve formě souborů.
V tomto případě se jedná o nahrávky ve formátu MP3, hlavní části jsou
však soubory ve formátu TEITOK, které obsahují všechny přepisy a anotace
včetně metadat. Tyto soubory jsou navzájem provázány s odpovídajícími
nahrávkami.

\hypertarget{struktura-souborux16f-teitok}{%
\subsubsection{Struktura souborů
TEITOK}\label{struktura-souborux16f-teitok}}

Formát TEITOK je formát XML, který plně odpovídá standardu
\href{https://www.tei-c.org/}{Text Encoding Initiative (TEI)}, avšak s
mírně odlišným přístupem k tokenizaci. Struktura TEITOK souborů v naší
databázi je následující:

\hypertarget{hlaviux10dka-s-metadaty-teiheader}{%
\paragraph{\texorpdfstring{Hlavička s metadaty
\texttt{\textless{}teiHeader\textgreater{}}}{Hlavička s metadaty \textless teiHeader\textgreater{}}}\label{hlaviux10dka-s-metadaty-teiheader}}

\begin{enumerate}
\def\labelenumi{\arabic{enumi}.}
\tightlist
\item
  \textbf{\texttt{\textless{}fileDesc\textgreater{}}} -- Popis souboru

  \begin{itemize}
  \tightlist
  \item
    \textbf{\texttt{\textless{}titleStmt\textgreater{}}}: Obsahuje název
    souboru a informace o autorech a anotátorech.
  \item
    \textbf{\texttt{\textless{}editionStmt\textgreater{}}}: Obsahuje
    číslo verze.
  \item
    \textbf{\texttt{\textless{}publicationStmt\textgreater{}}}:
    Publikační detaily, jako je vydavatel, datum vydání a licence.
  \item
    \textbf{\texttt{\textless{}sourceDesc\textgreater{}}}: Popis
    zdrojové nahrávky a odkaz na ni.
  \end{itemize}
\item
  \textbf{\texttt{\textless{}encodingDesc\textgreater{}}} -- Popis
  kódování

  \begin{itemize}
  \tightlist
  \item
    \textbf{\texttt{\textless{}projectDesc\textgreater{}}}: Stručný
    popis projektu, v rámci něhož data vznikla.
  \item
    \textbf{\texttt{\textless{}annotationDecl\textgreater{}}}: Detaily o
    jednotlivých krocích anotace (primární, revize, lingvistická
    anotace).
  \end{itemize}
\item
  \textbf{\texttt{\textless{}profileDesc\textgreater{}}} -- Profil textu

  \begin{itemize}
  \tightlist
  \item
    \textbf{\texttt{\textless{}langUsage\textgreater{}}}: Použitý jazyk
    (čeština).
  \item
    \textbf{\texttt{\textless{}textClass\textgreater{}}}: Metadata
    dokumentu:

    \begin{itemize}
    \tightlist
    \item
      \texttt{database}: Název databáze.
    \item
      \texttt{exam-id}: Identifikátor zkoušky.
    \item
      \texttt{cefr-level}: Úroveň podle SERR. Tato databáze obsahuje
      výhradně nahrávky zkoušek úrovně A2.
    \item
      \texttt{task-number}: Číslo úlohy.
    \item
      \texttt{preannot-source}: Zdroj předběžné anotace.
    \item
      \texttt{annotator}: Kód anotátora.
    \item
      \texttt{canonical}: Hodnota \texttt{1} značí kanonický přepis.
    \end{itemize}
  \end{itemize}
\end{enumerate}

\hypertarget{hlavnuxed-obsah-text}{%
\paragraph{\texorpdfstring{Hlavní obsah
\texttt{\textless{}text\textgreater{}}}{Hlavní obsah \textless text\textgreater{}}}\label{hlavnuxed-obsah-text}}

Sekce \texttt{\textless{}text\textgreater{}} obsahuje jednotlivé úseky
mluveného projevu strukturované pomocí elementů
\texttt{\textless{}u\textgreater{}}: -
\textbf{\texttt{\textless{}u\textgreater{}}}: Každý element
\texttt{\textless{}u\textgreater{}} reprezentuje úsek projevu a má
atributy: - \texttt{start} a \texttt{end}: Počáteční a koncový čas v
sekundách. - \texttt{who}: Mluvčí (např. ``EXAM\_1'' pro zkoušejícího a
``CAND\_1'' pro kandidáta). -
\textbf{\texttt{\textless{}s\textgreater{}}}: Každá věta je označena
elementem \texttt{\textless{}s\textgreater{}}. -
\textbf{\texttt{\textless{}tok\textgreater{}}}: Elementy tokenů, jejichž
atributy popisují lemma, slovní druh, morfologické rysy a syntaktický
vztah. - \textbf{\texttt{\textless{}anon/\textgreater{}}}: Anonymizovaný
úsek nahrávky. -
\textbf{\texttt{\textless{}gap\ reason="unintelligible"/\textgreater{}}}:
Nesrozumitelný úsek nahrávky.

\hypertarget{pux159uxedprava-souborux16f-teitok}{%
\subsubsection{Příprava souborů
TEITOK}\label{pux159uxedprava-souborux16f-teitok}}

Příprava souborů TEITOK probíhala v několika fázích:

\begin{enumerate}
\def\labelenumi{\arabic{enumi}.}
\tightlist
\item
  \textbf{Předběžná anotace}. V rámci výzkumu spojeného s vytvářením
  databáze jsme porovnávali přímou ruční anotaci s manuální post-editací
  výstupů systémů pro automatické rozpoznávání řeči. Manuální anotace
  tak může vycházet z automaticky připravené předběžné anotace. Zdroj
  předběžné anotace rozlišujeme pomocí atributu
  \texttt{preannot-source}, jehož hodnota může být:

  \begin{itemize}
  \tightlist
  \item
    \texttt{from\_scratch}: Kompletně manuální anotace, t.j. předběžná
    anotace je prázdná.
  \item
    \texttt{from\_whisperX}: Předběžná anotace získaná pomocí systému
    \href{https://github.com/m-bain/whisperX}{WhisperX}.
  \item
    \texttt{from\_mixed}: Předběžná anotace získaná náhodným
    kombinovaním výstupů čtyř systémů na úrovni replik.
  \end{itemize}
\end{enumerate}

Když předběžná anotace nebyla prázdná, převedli jsme ji do základní
verze formátu TEITOK. Na konci této fáze tak obsahovala přepisy
rozdělené do replik (elementy \texttt{\textless{}u\textgreater{}}),
přiřazení mluvčích k replikám (atribut \texttt{who}) a časové zarovnání
s nahrávkou (atributy \texttt{start} a \texttt{end}).

\begin{enumerate}
\def\labelenumi{\arabic{enumi}.}
\setcounter{enumi}{1}
\item
  \textbf{Manuální anotace}. Po nahrání souborů provedly zaškolené
  anotátorky manuální anotaci v prostředí TEITOK, během níž vytvářely
  nebo opravovaly přepisy, přiřazovaly mluvčí k replikám a pomocí
  časových značek zarovnávaly repliky s nahrávkou. Nahrávky byly
  anonymizovány v souladu s požadavky Ústavu jazykové a odborné přípravy
  Univerzity Karlovy (ÚJOP UK), který audionahrávky pro korpus poskytl.
  Některé anotátorky z opatrnosti anonymizovaly i údaje, které
  anonymizovány být nemusely (např. smyšlená jména osob).
\item
  \textbf{Revize}. Ruční kontrola manuálních anotací spoluautorkou
  databáze.
\item
  \textbf{Normalizace}. Automatická úprava přepisů, která odstraní
  odchylky ve jménech mluvčích, seřadí repliky podle počátečního času a
  přidělí replikám nové sekvenční ID.
\item
  \textbf{Rozdělení na úlohy a selekce}. Poskytovatel nahrávek (ÚJOP UK)
  povolil ke zveřejnění pouze vybrané úlohy. Ty jsme museli z nahrávek
  vystřihnout a upravit časové značky v přepisech, aby se zachovalo
  zarovnání replik v přepisu s nahrávkou. Pro stříhání nahrávky jsme
  použili nástroj \href{https://www.ffmpeg.org/}{FFmpeg}.
\item
  \textbf{Lingvistická anotace}. Až do této fáze nebyly repliky v
  přepisech dále strukturovány. V této fázi jsme text rozdělili na věty
  (element \texttt{\textless{}s\textgreater{}}) a následně věty na
  tokeny (elemety \texttt{\textless{}tok\textgreater{}}). Na úrovni
  tokenů jsou přepisy automaticky lingvisticky anotovány. Každému tokenu
  je přiděleno lemma (atribut \texttt{lemma}), jazykově specifická
  morfologická značka (atribut \texttt{xpos}), slovní druh a
  morfologické vlastnosti dle kategorizace projektu
  \href{https://universaldependencies.org/}{Universal Dependencies}
  (atributy \texttt{upos} a \texttt{feats}). Dále je každému tokenu
  přiřazen odkaz na ID rodiče podle pravidel závislostní syntaxe
  (atribut \texttt{head}) a typ závislosti tokenu ve vztahu k jeho
  rodiči (atribut \texttt{deprel}). Pro lingvistickou anotaci, včetně
  tokenizace, jsme použili nástroj
  \href{https://ufal.mff.cuni.cz/udpipe/2}{UDPipe 2}, konkrétně model
  \texttt{czech-pdt-ud-2.12-230717} pro češtinu. Ačkoli je možné
  provádět tokenizaci a automatickou lingvistickou anotaci přímo v
  prostředí TEITOK, my jsme tento proces realizovali samostatně. Důvodem
  je, že metoda tokenizace v prostředí TEITOK se liší od té, která je
  optimalizována pro UDPipe, což by mohlo způsobovat chyby při spojování
  těchto dvou kroků.
\item
  \textbf{Doplnění hlavičky TEI}. Na závěr jsme doplnili hlavičku podle
  všech dostupných metadat, aby odpovídala standardům TEI.
\end{enumerate}

Všechy nástroje a skripty (převážně v jazycích Python 3 a BASH) jsou k
dispozici ve \href{https://github.com/ufal/evaldio}{veřejném repozitáři
projektu} v adresáři \texttt{data\_preparation}.

\hypertarget{dotazovuxe1nuxed-vyhleduxe1vuxe1nuxed-a-filtrovuxe1nuxed}{%
\subsubsection{Dotazování, vyhledávání a
filtrování}\label{dotazovuxe1nuxed-vyhleduxe1vuxe1nuxed-a-filtrovuxe1nuxed}}

Rychlé dotazování, vyhledávání a filtrace jsou umožněny integrovaným
\href{https://cwb.sourceforge.io/files/CQP_Manual.pdf}{procesorem dotazů
CQP}, klíčovou komponentou sady nástrojů
\href{https://cwb.sourceforge.io/}{IMS Open Corpus Workbench (CWB)}. CQP
převádí korpusy ve formátu XML do binární podoby a efektivně je
indexuje. Dotazování v indexovaných korpusech probíhá pomocí jazyka
\href{https://www.cambridge.org/sketch/help/userguides/CQL\%20Help\%201.3.pdf}{CQL},
který je standardem v korpusové lingvistice. TEITOK také nabízí Query
builder, v němž může uživatel specifikovat dotaz vyplněním formuláře.
Výsledek dotazu vrácený z CQP je následně zpracován pomocí TEITOKu a
zobrazen uživateli v přehledné formě. Výsledky dotazů je možné stáhnout
ve formátu XML.

\end{document}
