% Options for packages loaded elsewhere
\PassOptionsToPackage{unicode}{hyperref}
\PassOptionsToPackage{hyphens}{url}
%
\documentclass[
]{article}
\usepackage{lmodern}
\usepackage{amssymb,amsmath}
\usepackage{ifxetex,ifluatex}
\ifnum 0\ifxetex 1\fi\ifluatex 1\fi=0 % if pdftex
  \usepackage[T1]{fontenc}
  \usepackage[utf8]{inputenc}
  \usepackage{textcomp} % provide euro and other symbols
\else % if luatex or xetex
  \usepackage{unicode-math}
  \defaultfontfeatures{Scale=MatchLowercase}
  \defaultfontfeatures[\rmfamily]{Ligatures=TeX,Scale=1}
\fi
% Use upquote if available, for straight quotes in verbatim environments
\IfFileExists{upquote.sty}{\usepackage{upquote}}{}
\IfFileExists{microtype.sty}{% use microtype if available
  \usepackage[]{microtype}
  \UseMicrotypeSet[protrusion]{basicmath} % disable protrusion for tt fonts
}{}
\makeatletter
\@ifundefined{KOMAClassName}{% if non-KOMA class
  \IfFileExists{parskip.sty}{%
    \usepackage{parskip}
  }{% else
    \setlength{\parindent}{0pt}
    \setlength{\parskip}{6pt plus 2pt minus 1pt}}
}{% if KOMA class
  \KOMAoptions{parskip=half}}
\makeatother
\usepackage{xcolor}
\IfFileExists{xurl.sty}{\usepackage{xurl}}{} % add URL line breaks if available
\IfFileExists{bookmark.sty}{\usepackage{bookmark}}{\usepackage{hyperref}}
\hypersetup{
  hidelinks,
  pdfcreator={LaTeX via pandoc}}
\urlstyle{same} % disable monospaced font for URLs
\usepackage{longtable,booktabs}
% Correct order of tables after \paragraph or \subparagraph
\usepackage{etoolbox}
\makeatletter
\patchcmd\longtable{\par}{\if@noskipsec\mbox{}\fi\par}{}{}
\makeatother
% Allow footnotes in longtable head/foot
\IfFileExists{footnotehyper.sty}{\usepackage{footnotehyper}}{\usepackage{footnote}}
\makesavenoteenv{longtable}
\setlength{\emergencystretch}{3em} % prevent overfull lines
\providecommand{\tightlist}{%
  \setlength{\itemsep}{0pt}\setlength{\parskip}{0pt}}
\setcounter{secnumdepth}{-\maxdimen} % remove section numbering

\author{}
\date{}

\begin{document}

\hypertarget{database-of-spoken-czech-as-a-foreign-language-permanent-residency-in-the-czech-republic}{%
\section{Database of Spoken Czech as a Foreign Language (Permanent
Residency in the Czech
Republic)}\label{database-of-spoken-czech-as-a-foreign-language-permanent-residency-in-the-czech-republic}}

Database of Spoken Czech as a Foreign Language (Permanent Residency in
the Czech Republic) is the language corpus of spoken performances by
non-native speakers of Czech focused on A2 level (according to the
CEFR), which is required for the granting of permanent residency in the
Czech Republic. It includes recordings capturing the oral part of the
\href{https://ujop.cuni.cz/UJOPEN-70.html?ujopcmsid=12:czech-language-certificate-exam-cce}{Czech
Language Certificate Exam}. The recordings consist of dialogues between
the examiner (a native speaker) and the candidate (a non-native
speaker). In addition to the recordings, the corpus also contains their
transcriptions, which are richly linguistically annotated. Some
recordings are accompanied by multiple transcriptions from different
annotators, allowing for comparisons of various transcripts of the same
recording and evaluations of the degree of consistency in converting
spoken language into written text.

The corpus is published as a specialized public database aimed at
providing a structured and easily accessible source of authentic spoken
data for linguists, educators, students, the scientific community, and
the general public.

The corpus was created at the \href{https://ufal.mff.cuni.cz/}{Institute
of Formal and Applied Linguistics at the Faculty of Mathematics and
Physics, Charles University} to support teaching, research, and
assessment of language competence among non-native speakers of Czech as
part of the project
\href{https://ufal.mff.cuni.cz/automated-speech-scoring-czech}{\emph{Automated
Speech Scoring in Czech}}. Audio recordings were provided by the
\href{https://ujop.cuni.cz/UJOPEN-1.html}{Institute for Language and
Preparatory Studies, Charles University} (ujop.cuni.cz).

\hypertarget{statistics}{%
\subsection{Statistics}\label{statistics}}

The database contains 63 recordings. It captures 41 exams and the same
number of non-native speakers. The total length of all recordings is 3h
18min 40s. The table below shows the transcription statistics, with one
canonical transcription selected for each recording.

\begin{longtable}[]{@{}lrr@{}}
\toprule
& All & Canonical\tabularnewline
\midrule
\endhead
Files & 106 & 63\tabularnewline
Utterances & 4,773 & 2,888\tabularnewline
Tokens & 33,267 & 20,035\tabularnewline
\bottomrule
\end{longtable}

\hypertarget{documentation}{%
\subsection{Documentation}\label{documentation}}

\begin{itemize}
\tightlist
\item
  \href{USER_MANUAL.md}{User Manual}
\item
  \href{TECH_DOC.md}{Technical Documentation}
\end{itemize}

\hypertarget{license}{%
\subsection{License}\label{license}}

The corpus is published under the CC BY-NC-SA 4.0 license.

\hypertarget{acknowledgment}{%
\subsection{Acknowledgment}\label{acknowledgment}}

The database was funded by the Programme to Support Applied Research in
the Area of the National and Cultural Identity for the Years 2023 to
2030 (NAKI III) of the Ministry of Culture of the Czech Republic within
the project \emph{Automated Speech Scoring in Czech} (DH23P03OVV037).

\hypertarget{special-thanks}{%
\subsection{Special Thanks}\label{special-thanks}}

The authors of the database sincerely thank PhDr. Pavel Pečený, Ph.D.,
from the Institute for Language and Preparatory Studies, Charles
University for providing audio data.

\hypertarget{how-to-cite}{%
\subsection{How to Cite}\label{how-to-cite}}

Rysová Kateřina, Novák Michal, Rysová Magdaléna, Polák Peter, Bojar
Ondřej: \emph{Database of Spoken Czech as a Foreign Language (Permanent
Residency in the Czech Republic)}. Institute of Formal and Applied
Linguistics MFF UK, Prague 2024. Available from WWW
https://lindat.mff.cuni.cz/services/teitok-live/evaldio/en/index.php?action=db\_residency.

\end{document}
