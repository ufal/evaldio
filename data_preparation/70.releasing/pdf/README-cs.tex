% Options for packages loaded elsewhere
\PassOptionsToPackage{unicode}{hyperref}
\PassOptionsToPackage{hyphens}{url}
%
\documentclass[
]{article}
\usepackage{lmodern}
\usepackage{amssymb,amsmath}
\usepackage{ifxetex,ifluatex}
\ifnum 0\ifxetex 1\fi\ifluatex 1\fi=0 % if pdftex
  \usepackage[T1]{fontenc}
  \usepackage[utf8]{inputenc}
  \usepackage{textcomp} % provide euro and other symbols
\else % if luatex or xetex
  \usepackage{unicode-math}
  \defaultfontfeatures{Scale=MatchLowercase}
  \defaultfontfeatures[\rmfamily]{Ligatures=TeX,Scale=1}
\fi
% Use upquote if available, for straight quotes in verbatim environments
\IfFileExists{upquote.sty}{\usepackage{upquote}}{}
\IfFileExists{microtype.sty}{% use microtype if available
  \usepackage[]{microtype}
  \UseMicrotypeSet[protrusion]{basicmath} % disable protrusion for tt fonts
}{}
\makeatletter
\@ifundefined{KOMAClassName}{% if non-KOMA class
  \IfFileExists{parskip.sty}{%
    \usepackage{parskip}
  }{% else
    \setlength{\parindent}{0pt}
    \setlength{\parskip}{6pt plus 2pt minus 1pt}}
}{% if KOMA class
  \KOMAoptions{parskip=half}}
\makeatother
\usepackage{xcolor}
\IfFileExists{xurl.sty}{\usepackage{xurl}}{} % add URL line breaks if available
\IfFileExists{bookmark.sty}{\usepackage{bookmark}}{\usepackage{hyperref}}
\hypersetup{
  hidelinks,
  pdfcreator={LaTeX via pandoc}}
\urlstyle{same} % disable monospaced font for URLs
\usepackage{longtable,booktabs}
% Correct order of tables after \paragraph or \subparagraph
\usepackage{etoolbox}
\makeatletter
\patchcmd\longtable{\par}{\if@noskipsec\mbox{}\fi\par}{}{}
\makeatother
% Allow footnotes in longtable head/foot
\IfFileExists{footnotehyper.sty}{\usepackage{footnotehyper}}{\usepackage{footnote}}
\makesavenoteenv{longtable}
\setlength{\emergencystretch}{3em} % prevent overfull lines
\providecommand{\tightlist}{%
  \setlength{\itemsep}{0pt}\setlength{\parskip}{0pt}}
\setcounter{secnumdepth}{-\maxdimen} % remove section numbering

\author{}
\date{}

\begin{document}

\hypertarget{databuxe1ze-mluvenuxfdch-projevux16f-v-ux10deux161tinux11b-jako-cizuxedm-jazyce-trvaluxfd-pobyt-v-ux10dr}{%
\section{Databáze mluvených projevů v češtině jako cizím jazyce (trvalý
pobyt v
ČR)}\label{databuxe1ze-mluvenuxfdch-projevux16f-v-ux10deux161tinux11b-jako-cizuxedm-jazyce-trvaluxfd-pobyt-v-ux10dr}}

Databáze mluvených projevů v češtině jako cizím jazyce (trvalý pobyt v
ČR) je jazykový korpus mluvených projevů nerodilých mluvčích češtiny
zaměřený na jazykovou úroveň A2 (podle SERR), požadovanou pro udělení
trvalého pobytu v České republice. Obsahuje nahrávky zaznamenávající
ústní část \href{http://ujop.cuni.cz/cce}{Certifikované zkoušky z
češtiny pro cizince}. Nahrávky zahrnují dialogy mezi zkoušejícím
(rodilým mluvčím) a kandidátem zkoušky (nerodilým mluvčím). Kromě
nahrávek korpus obsahuje také jejich přepisy, které jsou opatřeny
bohatou lingvistickou anotací. K některým nahrávkám je připojeno více
přepisů od různých anotátorů, což umožňuje srovnání různých přepisů téže
nahrávky a vyhodnocení míry shody při převodu mluvené řeči do psaného
textu.

Korpus je zveřejněn jako specializovaná veřejná databáze s cílem
poskytnout strukturovaný a snadno přístupný zdroj autentických mluvených
dat pro lingvisty, pedagogy, studenty, vědeckou komunitu a širokou
veřejnost.

Jazykový korpus byl vytvořen v \href{https://ufal.mff.cuni.cz/}{Ústavu
formální a aplikované lingvistiky Matematicko-fyzikální fakulty
Univerzity Karlovy} za účelem podpory výuky, výzkumu a hodnocení
jazykové kompetence nerodilých mluvčích češtiny v rámci projektu
\href{https://ufal.mff.cuni.cz/automated-speech-scoring-czech}{\emph{Automatické
hodnocení mluveného projevu v češtině}}. Audionahrávky poskytl
\href{https://ujop.cuni.cz/}{Ústav jazykové a odborné přípravy
Univerzity Karlovy} (ujop.cuni.cz).

\hypertarget{statistiky}{%
\subsection{Statistiky}\label{statistiky}}

Databáze obsahuje 63 nahrávek. Zachycuje 41 zkoušek a stejný počet
nerodilých mluvčích. Celková délka všech nahrávek je 3h 18min 40s.
Tabulka níže ukazuje statistiky přepisů, přičemž pro každou nahrávku byl
vybrán právě jeden kanonický přepis.

\begin{longtable}[]{@{}lrr@{}}
\toprule
& Všechny & Kanonické\tabularnewline
\midrule
\endhead
Soubory & 106 & 63\tabularnewline
Repliky & 4 773 & 2 888\tabularnewline
Tokeny & 33 267 & 20 035\tabularnewline
\bottomrule
\end{longtable}

\hypertarget{dokumentace}{%
\subsection{Dokumentace}\label{dokumentace}}

\begin{itemize}
\tightlist
\item
  \href{USER_MANUAL-cs.md}{Uživatelská příručka}
\item
  \href{TECH_DOC-cs.md}{Technická dokumentace}
\end{itemize}

\hypertarget{licence}{%
\subsection{Licence}\label{licence}}

Korpus je zveřejněn pod licencí CC BY-NC-SA 4.0.

\hypertarget{financovuxe1nuxed}{%
\subsection{Financování}\label{financovuxe1nuxed}}

Vznik databáze byl financován z prostředků Programu na podporu
aplikovaného výzkumu v oblasti národní a kulturní identity na léta 2023
až 2030 (NAKI III) Ministerstva kultury ČR v rámci projektu
\emph{Automatické hodnocení mluveného projevu v češtině}
(DH23P03OVV037).

\hypertarget{podux11bkovuxe1nuxed}{%
\subsection{Poděkování}\label{podux11bkovuxe1nuxed}}

Autoři databáze srdečně děkují PhDr. Pavlovi Pečenému, Ph.D., z Ústavu
jazykové a odborné přípravy Univerzity Karlovy za poskytnutí audiodat.

\hypertarget{jak-citovat}{%
\subsection{Jak citovat}\label{jak-citovat}}

Rysová Kateřina, Novák Michal, Rysová Magdaléna, Polák Peter, Bojar
Ondřej: \emph{Databáze mluvených projevů v češtině jako cizím jazyce
(trvalý pobyt v ČR)}. Ústav formální a aplikované lingvistiky MFF UK,
Praha 2024. Dostupná z WWW
\url{https://lindat.mff.cuni.cz/services/teitok-live/evaldio/cs/index.php?action=db_residency}.

\end{document}
