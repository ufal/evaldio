% Options for packages loaded elsewhere
\PassOptionsToPackage{unicode}{hyperref}
\PassOptionsToPackage{hyphens}{url}
%
\documentclass[
]{article}
\usepackage{lmodern}
\usepackage{amssymb,amsmath}
\usepackage{ifxetex,ifluatex}
\ifnum 0\ifxetex 1\fi\ifluatex 1\fi=0 % if pdftex
  \usepackage[T1]{fontenc}
  \usepackage[utf8]{inputenc}
  \usepackage{textcomp} % provide euro and other symbols
\else % if luatex or xetex
  \usepackage{unicode-math}
  \defaultfontfeatures{Scale=MatchLowercase}
  \defaultfontfeatures[\rmfamily]{Ligatures=TeX,Scale=1}
\fi
% Use upquote if available, for straight quotes in verbatim environments
\IfFileExists{upquote.sty}{\usepackage{upquote}}{}
\IfFileExists{microtype.sty}{% use microtype if available
  \usepackage[]{microtype}
  \UseMicrotypeSet[protrusion]{basicmath} % disable protrusion for tt fonts
}{}
\makeatletter
\@ifundefined{KOMAClassName}{% if non-KOMA class
  \IfFileExists{parskip.sty}{%
    \usepackage{parskip}
  }{% else
    \setlength{\parindent}{0pt}
    \setlength{\parskip}{6pt plus 2pt minus 1pt}}
}{% if KOMA class
  \KOMAoptions{parskip=half}}
\makeatother
\usepackage{xcolor}
\IfFileExists{xurl.sty}{\usepackage{xurl}}{} % add URL line breaks if available
\IfFileExists{bookmark.sty}{\usepackage{bookmark}}{\usepackage{hyperref}}
\hypersetup{
  hidelinks,
  pdfcreator={LaTeX via pandoc}}
\urlstyle{same} % disable monospaced font for URLs
\setlength{\emergencystretch}{3em} % prevent overfull lines
\providecommand{\tightlist}{%
  \setlength{\itemsep}{0pt}\setlength{\parskip}{0pt}}
\setcounter{secnumdepth}{-\maxdimen} % remove section numbering

\author{}
\date{}

\begin{document}

\hypertarget{user-manual}{%
\section{User Manual}\label{user-manual}}

The basic functions of the database include browsing records with
various display options, filtering records by different categories, and
performing complex searches within the database content. The database
also allows users to download the entire corpus or selected records.

\hypertarget{browsing-records}{%
\subsection{Browsing Records}\label{browsing-records}}

Upon entering the corpus, all records (i.e., transcript files) stored in
the database are displayed in a clear table. For each transcript file,
the table shows, in addition to the file name, the level and identifier
of the exam, the task number, the source of the preliminary annotation,
the annotator's code, and information on whether the transcript for that
recording is canonical. The files in the table can be sorted by the
values in a selected column. Records can also be filtered based on any
substring in the file name by entering this substring in the ``Search''
text box located to the right above the table. Clicking on a specific
file will display that file.

\hypertarget{viewing-a-file}{%
\subsection{Viewing a File}\label{viewing-a-file}}

The database allows users to view the transcripts of individual turns
along with annotations and metadata, and to listen to the corresponding
audio recordings. The nature of the displayed information varies
according to the selected display mode, which can be switched at the
bottom of the page below the transcript.

\hypertarget{text-view-mode}{%
\subsubsection{Text View Mode}\label{text-view-mode}}

Text View is the basic display mode that appears upon opening a file. At
the top of the screen is a header with the title of the transcript and
selected metadata. The transcript itself is displayed at the bottom,
divided into turns. Each turn is marked with the speaker's identifier
(EXAM\_1 for the examiner and CAND\_1 for the candidate).

This mode also allows users to view automatic morphological annotation
and lemmatization. Hovering the cursor over a specific token will
display the corresponding annotation in context. To display a selected
attribute for all tokens in the transcript, controls located below the
header can be used, which include the following buttons: - PoS: Displays
parts of speech. - Tag: Shows morphological tags. - Features: Provides
detailed morphological information. - Lemma: Displays base forms of
words.

\hypertarget{waveform-view-mode}{%
\subsubsection{Waveform View Mode}\label{waveform-view-mode}}

At the top of the screen, there is an extended playback control for the
recording, which displays a signal graph (i.e., waveform). Below it, the
transcripts of individual turns are displayed. Clicking on a specific
turn will play that turn.

\hypertarget{dependencies-mode}{%
\subsubsection{Dependencies Mode}\label{dependencies-mode}}

This mode displays syntactic annotation. When clicking on a specific
turn, an automatically generated dependency tree is displayed, with
details available via mouse hover. In the upper right corner of the tree
is a $\equiv$ button for additional display options for the tree. It is
possible to arrange nodes by word order, display punctuation, or save an
image of the tree in SVG format.

\hypertarget{filtering-records-by-categories}{%
\subsection{Filtering Records by
Categories}\label{filtering-records-by-categories}}

By clicking on the \emph{Browse} button in the left main menu, users can
filter transcripts based on the values of individual categories. For
example, it is possible to display only a list of canonical transcripts
or transcripts from a specific annotator.

\hypertarget{searching}{%
\subsection{Searching}\label{searching}}

Searching within the corpus can be done on a page that appears after
clicking the \emph{Search} button in the left main menu. This page
allows users to enter queries in CQL (Corpus Query Language) format. For
example:

\begin{quote}
\texttt{{[}upos\ =\ "NUM.*"{]}\ {[}lemma\ =\ "otázka"{]}}

to find forms of the word \emph{otázka} that are preceded by a numeral.
\end{quote}

To facilitate searching, the TEITOK interface provides a query builder
tool. This tool allows users to easily define simple queries in CQL
through a form. Just click the \emph{Query builder} icon, define your
query, and then press the \emph{Create query} button, which inserts the
query into the CQL text box where it can be further edited if needed.

By default, TEITOK searches the entire corpus, which may contain
multiple transcripts for a single recording. If you want to search only
in the part of the corpus where each recording has only a single
associated transcript, you must restrict the search to so-called
canonical transcripts. For example:

\begin{quote}
\texttt{{[}lemma\ =\ "situace"{]}\ ::\ match.text\_canonical\ =\ "1"}

searches for the lemma \emph{situace} only in canonical transcripts.
\end{quote}

\hypertarget{downloading}{%
\subsection{Downloading}\label{downloading}}

The entire corpus, including recordings and documentation, can be
downloaded from the main menu on the left.

A specific transcript can be downloaded in \emph{Text view} mode by
clicking the \emph{Download XML} button located at the bottom of the
page.

\end{document}
